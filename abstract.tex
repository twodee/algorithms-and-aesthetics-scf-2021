Designers interact with digital representations of objects in two different ways. With direct manipulation, the designer transforms an object by clicking and dragging on the canvas with a mouse or other pointing device. With indirect manipulation, the designer writes code that generates the object when the code is executed. Tools often support only one of these manipulation styles, which reinforces the false dichotomy separating artists from programmers and aesthetics from algorithms. In this paper, we present Twoville, a design tool that embraces both manipulation styles simultaneously. Programmer-artists use both their algorithmic and aesthetic senses in Twoville to create design files for 2D fabrication devices. Edits made in the code editor change the object in the canvas, and edits made in the canvas change the code that produced the object. We describe Twoville and offer an informal and formative evaluation of it through the eyes of a young beta tester.
