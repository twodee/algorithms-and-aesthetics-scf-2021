\section{Introduction}
\label{section:introduction}

% Twoville is a programming language and development environment for composing vector graphics files for tools like vinyl and laser cutters and pen plotters. Its users write code to generate, transform, and combine geometric shapes that determine the cutting or drawing path of these fabrication tools. The output of a program is a physical artifact that has a life beyond its digital roots.

In our experience, many educational makerspace efforts start with the best of intentions but degenerate into passive consumerism. Rather than flexing their own creative muscles, users download and fabricate designs made by others. One explanation for this behavior is that designing artifacts takes time, accessible tools, and trained educators, none of which may be available in an educational setting constrained by inflexible curriculum and inadequate resources. Grants provide money for equipment and supplies, but little else. The availability of fabrication equipment has helped democratize the physical manufacturing of artifacts, but designing one's own artifacts is still difficult.

Twoville is our attempt to make computation and fabrication more accessible for students in our local community. It is a programming environment for creating vector graphics files that can be fed into 2D fabrication tools like vinyl and laser cutters, pen plotters, and embroidery machines. The software runs in the browser and is open source. Its primary audience is learners in educational makerspaces. It offers a low barrier to entry so that novices may quickly fabricate something of their own design. At the same time, the scope of objects that may be designed is expansive. Writing Twoville programs is a cognitive process in which learners draw upon computational and mathematical knowledge. It is our hope that teachers can incorporate Twoville into their classroom activities and still meet curricular expectations.

As we've developed the language, we have come to the unoriginal conclusion that not all design work can be effectively programmed. Code is an indirect interface, and artists often want more direct control driven by aesthetic feel. Therefore, we provide an interface that supports bidirectional editing. The programmer-artist can edit the code, which updates the output, or edit the output, which updates the code. Objects may be shaped both algorithmically and through mouse-based direct manipulation of parameters.

This experience report is an introduction to the Twoville language and development environment. In Section~\ref{section:related_work}, we examine prior work that Twoville builds on. In Section~\ref{section:language}, we describe the language and the motivations behind our design choices. In Section~\ref{section:case_study}, we share an informal evaluation of Twoville by examining it through the eyes of a young beta tester.
